\documentclass[9pt,twocolumn,twoside]{gsajnl}
% Use the documentclass option 'lineno' to view line numbers

\articletype{inv} % article type
% {inv} Investigation 
% {gs} Genomic Selection
% {goi} Genetics of Immunity 
% {gos} Genetics of Sex 
% {mp} Multiparental Populations

\title{TCGA RNA-seq data analysis in breast invasive carcinoma}

\author[$\ast$]{Garcia-Serrano, A}
\author[$\ast$,1]{Gilabert-Navarro, JF.}
\author[$\ast$]{Madsen-Choppi, LPN.}

\affil[$\ast$]{Universitat Pompeu Fabra}

\keywords{Breast Invasive Carcinoma; RNA-seq; TCGA Project; Bioinformatics; Differential Expression}

\runningtitle{TCGA RNA-seq data analysis in breast invasive carcinoma} % For use in the footer 

\correspondingauthor{Corresponding Author}

\begin{abstract}

\end{abstract}

\setboolean{displaycopyright}{true}

\begin{document}

\maketitle
\thispagestyle{firststyle}
\marginmark
\firstpagefootnote
\correspondingauthoraffiliation{JF Gilabert: joanfrancesc.gilabert01@estudiant.upf.edu}
\vspace{-11pt}%

\section*{Introduction}
Breast cancer is the most common malignant cancer affecting women and is the second leading cause of cancer death worldwide\citep{rosam}. This disease has more than 1,300,000 cases and 450,000 death each year around the world\citep{cangen}.
This disease is widely heterogeneous, having a large and diverse set of molecular, histological and clinical behaviours depending of the tumour\citep{rosam}. In addition, the response to specific treatments it is also very different between patients. For this reason, breast cancer was been classified in different subtypes in order to achieve a better understanding of these disease. Traditionally, the classification has been based on clinicopathological features such as tumor type and size, lymph node status and histological grade\citep{rosam}. Actually, nowadays this disease is an entity difficult to classify due to the wide range of classifiers that we can take into account: histological, immunopathological, transcriptional, genomic, miRNA-based, epigenetic, microenvironmental, macroenvironmental, longitudinal and other classifiers\citep{breast}. However, we have an actual classification based on simple molecular characteristics\citep{cangen}
\begin{itemize}
\item \textbf{Estrogen receptor (ER) positive:} The most numerous and diverse, with several genomic tests to assist in predicting outcomes for ER+ patients receiving endocrine therapy.
\item \textbf{ HER2 or ERBB2 amplified:} Great clinical success because of effective therapeutic targeting of HER2, which has led to intense efforts to characterize other DNA copy number aberrations.
\item \textbf{Triple negative:} Lacking expression of ER, progesterone receptor (PR) and HER2. It is also known as basal-like breast cancers, are a group with only chemotherapy options, and have an increased incidence in patients with germline BRCA1 mutations or of African ancestry.
\end{itemize}

The more frequently mutated genes in breast cancer are BRCA1, BRCA2, PALB2, ATM, TP53, PTEN, PIK3CA, AKT1, GATA3, CDH1, RB1, MLL3, MAP3K1, CDKN1B, between others\citep{acs}. If we focus in their functionalities, these genes are linked with DNA repair, control of cell cycle, apoptosis, cell proliferation and gorwth. BRCA1 and BRCA2 mutations are the most common cause of hereditary breast cancer\citep{acs}. In addition, women with these mutations also have higher risk of developing other cancers, mainly ovarian cancer\citep{acs}. In case of BRCA1 mutations, the risk compared with the population is about 60\% meanwhile in BRCA2 mutations is about 45\%\citep{acs}. 
\vspace{2 mm}

Anyway, as we describe above, each tumour has a high specific profile with a lot of different variables difficulting the establishment of simple classifiers. In the last decades, molecular knowledge advances have allowed to initialize personalized medicine. In this way, we can use targeted drugs to very specific tumour types with a high percentage of effectiveness. The main problem of this personalized medicine is the very reduced number of tumours in which we can observe a remission. This is due to the very high specificity of the treatments, useful only for a tumour with a concrete molecular characteristics. 

\vspace{2 mm}

In this way, new technologies focused not only in mRNA expression profiling, DNA copy number analysis and massively parallel sequencing but also in detecting abnormalities in DNA methylation, miRNA and protein expression provides a wider range of information\citep{cangen}. Therefore, we can use all these tools in order to get a deeper understanding about tumor molecular mechanisms resulting in advances towards personalized medicine.

\section*{Materials and Methods}
All the following analysis were performed with \verb+R Studio Software Version 0.99.489 (R version 3.3.0)+. In this work we have used the following packages: $SummarizedExperiment$, $edgeR$, $geneplotter$, $sva$, $limma$, $GOstats$, $org.Hs.eg.db$ and $xtable$.

\subsection*{Data Availability}
\vspace{2 mm}

Our RNA-seq data set was obtained from \textbf{The Cancer Genome Atlas (TCGA) Project}. The data sets of this Project are tables of RNA-seq counts generated by Rahman et al \citep{Rahman15112015} from the TCGA raw sequence read data using the $Rsubread / featureCounts$ pipeline for all data sets.
\vspace{2 mm}

We choose Breast Invasive Carcinoma RDS file in order to perform our analyses. This set is composed by 1119 tumoral tissue samples and 113 healthy tissue samples. In this work we focus in a particular subset. The subsetting criteria was the selection of paired data; we only use these samples which have both tumoral and healthy tissue from the same patient. In this way, we try to minimize the effect of inter-personal variability due to the differences in genetic background.
\vspace{2 mm}
  
So, first of all we filter our data by the common \verb+brc_patient_barcode+ and replace our original data by this subset containing 212 samples in total; 106 tumoral tissue samples and 106 healthy tissue samples. All the statistical analysis explained below were performed using this new data set.


\subsection*{Statistical Analysis} 
\vspace{2mm}

\textbf{Quality control}
\vspace{2mm}

Before starting with the differential expression analysis we need to ensure the quality of our data in order to avoid bias in our results and in consequence, the extraction of wrong conclusions. 
\vspace{2mm}

Expression levels were considered taking into account the $log_{2}$ of $CPM$ values of expression. The first analyses that we performed were observe gene expression distribution and filtering again by genes less expressed. We considered a cutoff of 1 $log CPM$ unit as a minimum value of expression to select genes being expressed across samples.
\vspace{2mm}

After that, we calculated the normalization factors on this new filtered data set using the TMM method implemented in the edgeR package and we generated the MA-plots both for normal and tumoral samples (see in $Supplementary Material$).
\vspace{2mm}

Batch effect identification tests were performed taking into account different elements of the TGCA barcode such as tissue source site, the center where the samples were processed, the plate, the sample vial and the portion analyte (molecular specimen extracted; total RNA or whole genome amplified for example). After consider the variables described above, we did the tests related with the hierarchical clustering and multidimensional scaling examining the samples by the tissue source site (TSS). In this part we use again $log CPM$ values with a higher prior count to moderate extreme fold-changes produced by low counts. Finally, we generate a dendogram of hierarchical clustering of samples by TSS (see on $Results$ section). 
\vspace{4mm}

\textbf{Differential expression analysis}
\vspace{2mm}

In this section we want to analyze how many genes are differentially expressed (DE) across normal and tumoral samples. In order to do that, we need to create a model to start analyzing our data set. We have created two models, a simple model that only considers the type of sample (tumor or not) and one that also takes into account the patient barcode. It is important to introduce this variable into our model because of we want to avoid bias produced by variability in genetic background across individuals. The models are implemented following the pipeline of the limma package. After creating the model and before the statistics, we will use the surrogate variable analysis (SVA) to account for unknown covariates. Finally, we can visualize our DE results with a volcano plot (see also in $Results$ section).
\vspace{4mm}

\textbf{Functional enrichment analysis}
\vspace{2mm}

The last part of our analysis is focused in the functional interpretation of our DE genes results. To do that, we focused on the Gene Ontology biological processes since we are interested in understanding which pathways are more affected across breast tumors. To perform this analysis we will use the packages $GOstats$ to obtain the information relative to Gene Ontology and link it with our microarray data, $org.Hs.eg.db$ to obtain the genome wide annotation for Homo Sapiens using Entrez Gene Identifiers and finally we use $xtable$ to generate the output results. First step is to create \verb+DEgenes2+ variable containing the differential expressed genes (p-value < 0.05) obtained previously, later we define \verb+geneUniverse+ with all the Entrez IDs of genes contained in $org.HS.eg.db$. Finally we assign these Entrez IDs to our \verb+DEgenes2+ variable and start with $hyperGTest(params)$ in order to perform the hypergeometric tests for GO association.
\vspace{2mm}

Finally we filter the results only considering GO terms with gene size and gene counts greater than 5, since those with size smaller than 5 are not so reliable. To end up with this part the final results were ordered by the Odds Ratio and exported into an html output for a better visualization (view in $Results$ section).

\section*{Results and Discussion}

\section*{Conclusions}

\section*{Acknowledgments}

\bibliography{biblio.bib}

\end{document}